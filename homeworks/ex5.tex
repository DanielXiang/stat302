\documentclass[12pt]{article}
\def\P{\mbox{P}}
\def\G{\Gamma}
\def\t{\theta}
\def\a{\alpha}
\def\E{\mbox{E}}
\def\Be{\mbox{Beta}}
\def\to{\tau_0}
\def\ti{\tau_1}
\def\train{\text{Training data}}

\parindent=0in
%\parskip=5mm
\pagestyle{empty}
\usepackage{amsmath}
\usepackage{enumerate,verbatim}
\usepackage{url}

\begin{document}

\begin{center}
{\bf
Bayesian Statistics

\smallskip

Exercises 5.
}
\smallskip

\end{center}

\bigskip

\begin{enumerate}

\item Let $x_1,\dots,x_n$ be independent and identically distributed $N(\mu,1)$. 
Assume a prior for $\mu$ that is $N(0,3^2)$.
Explain what the following R code does - and, particularly, explain i) the vectorization being performed in the function {\tt loglik},
and ii) the problems caused when computing the weights {\tt w1}, and how the program solves this problem for computing {\tt w2}.
Compare the numerical results with analytical calculations
for the same data $x$.
\begin{verbatim}
set.seed(111)
m=10000
n=1000
x = rnorm(n,0,1)

musamp = rnorm(m,0,3)

loglik = function(x,musamp){
  n=length(x)
  m = length(musamp)
  xx = rep(x,rep(m,n))
  ll.matrix=matrix(dnorm(xx,musamp,1,log=TRUE),nrow=m)
  rowSums(ll.matrix)
}

logl=loglik(x,musamp)
normalize=function(x){x/sum(x)}
w1 = normalize(exp(logl))
w2 = normalize(exp(logl-max(logl)))

sum(w1[musamp<0])
sum(w2[musamp<0])
sum(w2*musamp)
sum(w2*musamp^2) - sum(w2*musamp)^2
\end{verbatim}



\item Let $x_1,\dots,x_n$ be independent and identically distributed observations from the mixture
density
\begin{equation} \label{eqn:mix}
f_X(x; \pi_0, \mu_0,\mu_1,\tau_0,\tau_1) = \pi_0 N(x; \mu_0, 1/\tau_0) + (1-\pi_0) N(x; \mu_1,1/\tau_1)
\end{equation}
where $N(\cdot; \mu,1/\tau)$ denotes the density of a normal distribution with mean $\mu$ and variance $1/\tau$. 

Assume that the true values of the parameters are $\pi_0= 0.2, \mu_0 = 0, \mu_1 = 4, \tau_0=\tau_1=1$. 

\begin{enumerate}
\item The mixture model (\ref{eqn:mix}) can also be written using the following ``latent variable representation":
$z_i \sim \text{Bernoulli}(1-\pi_0)$; $x_i | z_i =k \sim N(\mu_k, 1/\tau_k)$.
This is called the latent variable representation because $z_i$ is unobserved (latent) so
the probability density of the $x_i$ requires you to marginalize $p(x_i,z_i)$ over $z_i$.
Show how marginalizing the joint distribution over $z_i$ yields the 
density (\ref{eqn:mix}) for $x_i$.
\item Simulate 1000 observations from this mixture distribution.  (Make sure to set a seed so that your
results are reproducible.) Plot a histogram of the results. [Hint: simulating from a mixture 
can be done by exploiting the latent variable representation.]
\item Write a function to compute the log likelihood for a vector $x$ at any given $\pi_0,\mu_0,\mu_1,\tau_0,\tau_1$.
(You might also like to vectorize this function to allow the log-likelihood to be efficiently 
computed for any vectors of these parameters, analogous to the code in Q1)
\item Assume now that you know the true values of the means and precisions ($\mu_0,\mu_1,\tau_0,\tau_1$) and
wish to estimate $\pi_0$.
Assume that your prior is
\begin{equation}
\pi_0 \sim \Be(0.5,0.5)
\end{equation}
\begin{enumerate}[i)]
\item Write a function that uses important sampling, with the prior distribution as your importance sampling
distribution, to obtain an approximation to the posterior distribution of $\pi_0$. Apply this to your simulated data set (created in part b) to obtain an approximate
posterior mean and an approximate $90\%$ posterior CI for $\pi_0$. (Unless you are unlucky the posterior CI should cover the true value of $\pi_0$! If not, try again with a new seed, to check
whether there might be a problem with your implementation.)
\item Use your log-likelihood function to compute the
log-likelihood on a grid of values of $\pi_0$ from 0 to 1. Use this grid-based approach to obtain a discrete approximation to the
posterior distribution for $\pi_0$. Again, obtain a posterior mean for $\pi_0$ and a $90\%$ posterior CI. 

\item Use your answer to the previous part to develop an improved importance sampling function. (Note that, to satisfy the requirements
that the support of $q$ contains the support of $p$, the importance sampling
function must be continuous, so you have to convert your discrete approximation to a continuous approximation.) Use this
improved IS function to obtain another estimate for the posterior mean and an approximate $90\%$ posterior CI for $\pi_0$.
Comment on how the variance of the importance weights differs between this IS function and using the prior as an IS function.
\end{enumerate}
\end{enumerate}

\item Repeat part d) of question 1, considering the precisions $\tau_0,\tau_1$ also to be unknown with independent priors
\begin{equation}
\tau_k \sim \Gamma(\text{shape=0.1},\text{rate=0.1}).
\end{equation}
[Hint 1: the likelihood will now of course be $p(x_1,\dots,x_n | \pi_0,\tau_0,\tau_1)$. 
Hint 2: when considering a grid of values of $\tau$ you probably want to work with an equal-spaced grid on $\log(\tau)$ in some finite range.]

Do you feel that any of the three methods you try are able to get reliable estimates of the joint posterior distribution in this case? Give reasons
for your answer.

\item Repeat part d) of question 1, considering the precisions $\tau_0,\tau_1$ unknown with prior as above, and means $\mu_0,\mu_1$ to be unknown, with 
independent priors
\begin{equation}
\mu_k \sim N(0,3^2).
\end{equation}
(NOTE: remember that in R the normal functions are parameterized in terms of the standard deviation -- here 3 -- not the variance.)

Do you feel that any of the three methods you try  are able to get reliable estimates of the joint posterior distribution in this case? Give reasons
for your answer.  Comment on differences or similarities between the posterior distributions of $\mu_1$ and $\mu_2$.
 
\end{enumerate}


\end{document}



