\documentclass[12pt]{article}
\def\P{\mbox{P}}
\def\G{\Gamma}
\def\t{\theta}
\def\a{\alpha}
\def\E{\mbox{E}}
\parindent=0in
%\parskip=5mm
\pagestyle{empty}
%\usepackage{chicago}
\usepackage{url}
\usepackage{natbib}

\begin{document}

\begin{center}
{\bf
Bayesian Statistics}

\smallskip

\end{center}

\bigskip

Note: here the Gamma($m,\lambda$) distribution refers
to the distribution with density $p(x) \propto x^{m-1} \exp(-\lambda x)$

\begin{enumerate}

\item Read the article at
\url{http://www.nytimes.com/2011/01/11/science/11esp.html}. The following is a quotation from this article:
\begin{quote}
Consider the following experiment. Suppose there was reason to believe that a coin was slightly weighted toward heads. In a test, the coin comes up heads 527 times out of 1,000.
Is this significant evidence that the coin is weighted?
Classical analysis says yes. With a fair coin, the chances of getting 527 or more heads in 1,000 flips is less than 1 in 20, or 5 percent, the conventional cutoff. To put it another way: the experiment finds evidence of a weighted coin �with 95 percent confidence.�
Yet many statisticians do not buy it. One in 20 is the probability of getting any number of heads above 526 in 1,000 throws. That is, it is the sum of the probability of flipping 527, the probability of flipping 528, 529 and so on.
But the experiment did not find all of the numbers in that range; it found just one � 527. It is thus more accurate, these experts say, to calculate the probability of getting that one number � 527 � if the coin is weighted, and compare it with the probability of getting the same number if the coin is fair.
Statisticians can show that this ratio cannot be higher than about 4 to 1, according to Paul Speckman, a statistician, who, with Jeff Rouder, a psychologist, provided the example. 
\end{quote}

\begin{enumerate}
\item[i)] Formalize and confirm the statement ``Statisticians can show that this ratio cannot be higher than about 4 to 1". 
\item[ii)] In the above setting, if your prior probability that the coin is weighted is 0.5, what is the largest your posterior probability (that the coin is weighted) could be?
\end{enumerate}


\item Mr. Rubin has determined that his utility function for a change in fortune on the interval $r \in [-100,500]$ is $$U(r) = 0.62 \log (0.004r + 1).$$
\begin{enumerate}
\item[a)] He is offered a choice between \$100 and the chance to participate in a gamble wherein he wins \$0 with probability $2/3$ and \$500 with probability $1/3$. Which should he choose?
\item[b)] Suppose he is offered instead a chance to {\it pay} \$100 to participate in the gamble. Should he do it?
\end{enumerate}

\item Consider $x_1, x_2, \ldots, x_n$ i.i.d.$\sim$ Poisson($\mu$).  Suppose the prior distribution on $\mu$ is Gamma($m, \lambda$).  
\begin{enumerate}
\item Derive,  analytically, the posterior distribution for $\mu$. Explain the role of conjugacy here. Show that this posterior distribution depends on the data only through the sample mean $\bar{x}$ (and $n$). 
\item Plot the posterior distribution for $\mu$ given data with $\bar{x}=2.8, n=10$ with the prior $m=1,\lambda=1$. Compute the posterior mean, the posterior median, 
and give a 90\% Credible interval for $\mu$ (give code where you use R).
\item The posterior distribution can be interpreted in terms of ``long-run" frequency along the follow lines: 
if one repeated a certain simulation experiment, many times, then of the times we see ``data like this"
the posterior distribution tells us how often the parameter would fall in any given region.
Make this statement more explicit (ie say what we mean by ``a certain simulation experiment" and ``data like this") and illustrate it by running
the simulation. (Use the data and prior above). Provide your code as well as commentary.
\end{enumerate}

\item  Suppose that, given $\sigma$, $x$ is Normal with mean 0 and variance $\sigma^2$, and that the prior for $\sigma$ is such that $1/\sigma^2$ has a Gamma$(m,l)$ distribution.  Find the posterior distribution of $\sigma^2$.
[Hint: the algebra is easier if you compute the posterior for the ``precision", $\tau:=1/\sigma^2$, rather than for the variance $\sigma^2$. For this reason
Bayesian analyses often consider the precision rather than the variance.]

\item In Homework 1 you were asked to complete the exercise in the commented
text at the end of \verb|exercises/seeb/train_test.R|. 
\begin{enumerate}
\item Swap your code and results
with someone else from the class (henceforth referred to as your ``partner"). 
[Your partner should be someone with whom you have not previously compared 
code/results for this question - if you need help identifying a partner, let me know.]
\item Run your partner's code yourself, and check you get the same answer that they report.
If not, communicate with your partner to resolve the difference, and report the outcome.
\item Compare the results of your code with the results of your partner's code.
Are they the same? If not, communicate with your partner to resolve the difference, and report the outcome.
\item  Compare the way that your partner tackled the problem with the way
you tackled the problem. Comment on any similarities and differences. 
Is their code easier to read or harder to read than yours? Did they
use any R tricks or functions that you were unfamiliar with? If possible improve your 
code (and correct it if necessary) as a result of what you have learned. Hand in your improved code and its output.
\end{enumerate}

\end{enumerate}

\bibliographystyle{chicago}
\bibliography{/Users/stephens/Documents/mainbib}

\end{document}



