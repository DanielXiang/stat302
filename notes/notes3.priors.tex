\documentclass[12pt]{article}
\def \baselinestretch{1.2}
\def\ni{\noindent}
\def\Be{\text{Be}}

\usepackage{amsmath,url,enumerate}

\textwidth=17.0cm
\oddsidemargin 3cm
\evensidemargin 3cm

\topmargin=-1.2cm
\columnsep=1.0cm
\textheight=21.8cm
\hoffset=-3.2cm
\renewcommand{\baselinestretch}{1.2}
\setlength{\parindent}{0.3in}
\setlength{\parskip}{0.1in}

\setcounter{secnumdepth}{2}


\setcounter{section}{2}

\begin{document}

\section{Prior Distributions}

There are broadly two different formal approaches to specification of prior distributions:
\begin{enumerate}
\item Subjective specification: the prior is chosen by making a serious attempt to quantify a state of knowledge prior to performing the experiment. This is perhaps the ``purest" form of Bayesian analysis. This approach is most often used when the
prior actually contains non-trivial information that helps inform the analysis.
\item Objective specification: the prior is chosen to satisfy some objective criterion. 
Examples of criterion that are sometimes used are:
\begin{enumerate}
\item the conclusions of the analysis (i.e. posterior distributions of quantities of interest) should be invariant to scale of measurement (e.g. multiplying all data by a constant
as would happen if we switched from inches to centimeters). 
\item more generally, the conclusions of the analysis should be invariant to choice of parameterization. (Jeffrey's priors are the most common example.)
\item  the resulting posterior should have good frequentist properties, such as the correct (asymptotic) frequentist coverage for 95\% CIs. (``matching priors").
\end{enumerate}
\end{enumerate}
In both cases some consideration may be
given to computational convenience. For example,
one might take a subjective approach, but restricting
oneself to conjugate priors.

Usually in science the word ``subjective" is viewed as bad or suspicious, and ``objective" is viewed as good. But in this case it is far from as simple as this.
It might be fair to say that it is a subjective question whether it is better to use subjective or objective priors. And it would certainly be fair
to say there are many different objective priors, so which one you might use would be a subjective choice. Ultimately both approaches
can be useful, and indeed they are not mutually exclusive. For example, I might use an Objective specification for parameters about which I do not
have much subjective information, and subjective priors for parameters where I have specific and easily-quantified prior information I want to capture.

Although these are the two types of prior that have been studied and written about, in practical applications people often use priors that are not really either
subjective or objective. That is no serious attempt has been made to capture prior knowledge, but neither are they formally chosen to satisfy a stated objective criterion.
Indeed in realistic situations involving more than a few parameters it can be essentially impossible to perform an analysis that is truly subjective (because
of the difficulty of accurately quantifying ones prior beliefs) or objective
 (objective priors are only well developed for a relatively simple range of examples, such as inference for a normal mean...).

This may sound pretty bad, but in practice this generally works out fine in situations where i) the data are reasonably informative about the parameter in question, and ii) the prior information
available is relatively weak. In such cases the posterior is relatively insensitive to the prior chosen, and will closely approximate
what would have been obtained if one had done a serious and careful subjective analysis. 
%the prior isn't actually claimed to reflect the full state of knowledge prior to the experiment, but is chosen to have the property that the resulting {\it posterior} is nonetheless expected to be a very good approximation to what would have been obtained if a full subjective analysis had been undertaken. 
We saw an example of this in Savage's potato: using a very flat prior on the potato weight is not a sensible reflection of the ``true" prior state of knowledge, but -- as Savage argues -- it will result in a similar posterior, provided that the ``true" prior is approximately flat in the area with appreciable likelihood (and nowhere else much larger). 

In practice, the main challenge in prior specification is avoiding specifying a prior that is both i) unrealistic and ii) has a severe effect on the inference.
This is sometimes harder than expected. 

Note also there are many other terms people use for priors, some of which may be useful, others may not.
\begin{enumerate}
\item ``non-informative" priors. This isn't a useful term. No prior is ``non-informative". It isn't clear what it means. 
\item ``flat" priors. This usually means $p(\theta)$ is a constant, but you might as well say that explicitly.
\item ``diffuse" priors - this is a qualitative description, usually used to indicate that no serious attempt was made to
quantify subjective information, and usually implying that the author believes the results will be relatively insensitive to this choice.
\item reference prior: a specific type of objective prior due to Bernardo and co-workers.
\item Data-dependent priors: a prior distribution that has been chosen
after ``peeking" at the data. This sounds like a bad thing, and it can be. Sometimes it works out OK. You have to be careful!
\end{enumerate}


What about performing analyses using more than one prior? Strict subjective Bayesians could argue that one should simply do the analysis with
the prior distribution that reflects your prior knowledge/uncertainty.
However, it seems to me perfectly reasonable to 
ask whether individuals with different prior beliefs will come to different conclusions. Thus, in situations
where it is plausible that different individuals might have quite different prior beliefs it seems reasonable to report results of multiple analyses using different priors.

A question for thought/discussion:
Suppose that an Objective prior specification procedure resulted in a posterior distribution that was
inconsistent with your prior information. Would you be happy to use it anyway? What would the resulting posterior distributions mean?
What does a 95\% credible interval computed from an ``objective prior" mean? (For a subjective
prior, it has a relatively straightforward interpretation as representing a certain state of uncertainty in the parameter after viewing data.)



\subsection{Subjective Determination of the Prior}

In the subjective Bayesian world, the prior distribution is intended
to capture the information available about the parameter before the
data is observed.  We have discussed how use of probability distributions to represent uncertainty,
and updating beliefs via Bayes Theorem, leads to coherent, quantitative,
belief systems. However, there remains the fact that specifying a 
prior distribution that encapsulates (even approximately) the appropriate beliefs can be challenging.  

When the parameter space is discrete, the specification of a prior distribution boils down to simply comparing a finite number of options.
However, if the number of options is sufficiently
large it will be impractical to actually compare them all, in which case it is usually helpful to make use of parametric models or make simplifying assumptions (see for example the variable selection in regression example below). Similarly,
when the parameter space is continuous, it is usually helpful to restrict oneself to a particular parametric form, and then choose the parameters 
so that the resulting prior distribution approximates prior beliefs, perhaps by
  matching moments or (for more robustness) percentiles to prior assessments (see for example the ``relative risk" example below). 

Ultimately there is no ``right" approach to subjective prior specification. The following examples are intended to illustrate some of the issues one
may encounter in practice.

\underline{Examples:}
  
\begin{enumerate}
  \item Law enforcement agencies are interested in classifying African Elephant ivory as being from Forest or Savanna elephants (two subspecies of African elephant). What is your prior that an illegally-poached tusk is from Forest? Note that, of course, there is no ``right" answer to this question. Let's consider a series of possibilities.
  \begin{itemize}
  \item If we knew nothing more about the problem we might say 0.5 for each. (One possible argument for this is ``symmetry", which could
  be seen as an ``Objective" argument. Alternatively we could see it as a directly representation of our subjective knowledge.)
  \item Suppose that you had been involved in this kind of problem for a long time, and in your extensive experience, two thirds of illegally-poached tusfks
  were from forest elephants. Then you might use a prior $\Pr(\text{forest}) = 2/3$ and $\Pr(\text{savanna}) = 1/3$.
  Indeed, this might seem overwhelmingly an obvious thing to do. It may even seem ``objective". Certainly it is ``data driven". However, note
  that actually this prior is still making a (subjective) assumption: that the tusk in hand is somehow ``comparable with" or ``similar to" the tusks
  you have come across in the past. (We will better formalize this notion when we look at ``exchangeability" later.)  So...
  \item ... suppose that although two-thirds of tusks you have  come across in the past were from forest elephants, this particular tusk came
  from a collection of tusks, all seized at the same time, in a single shipment, and all bearing a similar distinctive red marking - and the other tusks in this shipment
  all turned out to be savanna tusks. In this case it would seem not unreasonable to assume a prior for this tusk that favors savanna. (Again, this will
  be seen to relate to exchangeability assumptions.)
  \item On the other hand, even given all the above, we might prefer to perform our analysis initially with the 50-50 prior, to see what the DNA data have
  to add to any of these other ``prior" sources of information. Alternatively, and perhaps preferably in this situation, we might rather report the results
  of the analysis as a Bayes Factor, since this has a direct interpretation in terms of how the DNA data should modify prior beliefs.
  For example, if an analysis with a 50-50 prior concluded that the tusk was, with probability 0.99, from savannah, and the police (or someone else) were skeptical of this conclusion, then we might ask the question, ``how strong would the prior belief in the forest have to be to outweigh the data?" [{\it Exercise!}] Note that this might be a general reason to consider priors that do not necessarily represent our actual prior beliefs: to convince a skeptic! In this example, Bayes Factors come in handy for seeing how the prior and data combine.
\end{itemize}  
  
  \def\fvec{(f_{CC},f_{CG},f_{GG})}
  
  \item At a C/G single nucleotide polymorphism (SNP), there are three possible genotypes (``classes") for each individual: CC,CG and GG. 
 Consider specifying a prior distribution for the  frequencies $f=\fvec$ for a sample of individuals from China.
 \begin{itemize}
 \item Someone who knew nothing more about this might start with a uniform prior on $f$ (subject to the constraint that $f_{CC}+f_{CG}+f_{GG}=1$). Indeed, even someone who knows quite a bit more might use this prior if they had lots of data from which they were about to compute a posterior, and figured that it was not worth worrying too much about more careful prior specification. (In the first case one might view this as a subjective prior; in the second a ``semi-subjective" prior).
 \item Someone who knows a bit about population genetics might decide to use the assumption of ``Hardy--Weinberg Equilibrium" (HWE), which is effectively the assumption that the two alleles carried by each individual are independent draws from some distribution. Thus, if $f_C$ denotes the overall frequency of $C$ in the population, then $f=(f_C^2, 2f_C(1-f_C),(1-f_C)^2)$. This assumption effectively reduces the vector parameter $f$ to a single parameter $f_C$, so
 putting a prior on $f$ can now be achieved by putting
 a prior on $f_C$. Perhaps we might choose a uniform prior on $[0,1)$ for $f_C$.
  \item Although HWE is a pretty reasonable assumption for many natural populations, there are many reasons that there might be small (or even perhaps large)
  deviations from HWE in practice. In particular, if individuals tend, even slightly to preferentially mate with others who are more ``genetically similar" then
  there will be an excess of CC and GG genotypes compared with expectations under HWE.
   A common way to allow for this kind of potential deviation from HWE is to introduce a new parameter, $\xi$ say, with $\xi \in [0,1]$, and
 $f=(1-\xi)(f_C^2, 2f_C(1-f_C),(1-f_C)^2) + \xi(f_C,0,1-f_C)$. Note that $\xi=0$ corresponds to HWE, while $\xi>0$ corresponds to an ``excess" of $CC$ and $GG$ genotypes
 compared with those expected under HWE. We might then assume independent uniform priors on $\xi$ and $f$. Or perhaps we should use a prior
 on $\xi$ that favors values near 0 (since we know that HWE often holds well in practice). Eg $\xi \sim Be(1,10)$.    
 \item Someone who knows a bit more about population genetics might know that, both from mathematical models and in actual data, allele frequencies like $f_C$ tend not to be uniform on $[0,1)$, but more skewed towards the ends 0 and 1. Perhaps they would use a Beta(0.5,0.5) or Beta(0.1,0.1) distribution for $f_C$. Better still, perhaps they would go out and gather frequency data on, say, all known SNPs in Chinese, and fit a Beta$(\alpha, \beta)$ distribution to those data,
 and use that as a prior for this new SNP. (Note that this strategy is ``borrowing information" across SNPs.)
 \item  Finally, suppose that from a large data set on Japanese individuals we know that the frequencies of the three types in Japan are $g=(g_{CC},g_{CG},g_{GG})$. Perhaps we could
 assume a Dirichlet prior for $f$ (this is the
 conjugate prior for multinomial sampling) in which $E(f)=g$. Thus $f \sim Dir(\alpha g_{CC},\alpha g_{CG}, \alpha g_{GG})$, where $\alpha$ controls how similar we expect $f$ to be to $g$, which 
 perhaps again we could get some idea from using data on other SNPs. (Note that this strategy
 is ``borrowing information" across both SNPs and populations).
   \end{itemize}
 Note that the above example illustrates the blurring between ``prior assumptions" and ``modeling assumptions". Here, we could describe the assumption of HWE as either
 a ``prior assumption" or a ``modeling assumption". In the first case we think of it as encoding a prior on $f=\fvec$, in the second we might say the
 likelihood has a single parameter $f_C$, rather than the full two-parameter parameterization in $\fvec$.
 Of course, whichever way you describe it, the resulting inference is identical.
 In fact, it turns out to be impossible, logically, to distinguish ``prior assumptions" from ``modeling assumptions". 
 So, for example, if we assume a normal distribution for the data when formulating a likelihood, this is effectively also prior assumption.
(A prior assumption that, with probability 1, the data come from a normal distribution). This may seem a trivial point, but in fact it is quite important
to remember that {\it any} analysis is making (almost inevitably subjective) assumptions. 

It is also interesting to note that people appear to be hard-wired to be more comfortable with direct assumptions than with ``fuzzy" priors.
For example, if I submit an analysis that starts ``We assume HWE holds,...." then provided this assumption seems somewhat reasonable in the
setting I am in then most readers or reviewers would not even think to object. However, suppose instead I say
``We assume that $f=(1-\xi)(f_C^2, 2f_C(1-f_C),(1-f_C)^2) + \xi(f_C,0,1-f_C)$,
with a mixture prior on $\xi \sim 0.5 \delta_0 + 0.5 \Be(0.5,0.5)$ where $\delta_0$ denotes a point mass on 0.". This prior assumption is
less restrictive than assuming HWE: with probability 0.5 HWE holds, and with probability 0.5 we allow some deviation from HWE. But
the natural initial reaction to this statement is to see it as a {\it stronger} assumption than HWE, and to ask ``where on earth does that prior come from"?
Similarly, if I say I assume that my data are normally distributed, then readers will generally not think to question the assumption. But if I say that I assume
they are $t$ distributed on $\nu$ degrees of freedom, with a prior on $\nu$ that is $p(\nu)= 0.5\delta_\infty + 0.5 \text{Unif}\{1,2,4,8,16,32,64\}$,
then again it is very natural to ask where this assumption came from (even though it is, in many ways, a weaker assumption than normality).
h    
 % BVSR EXAMPLE? \item Suppose that in a regression model, 
  \item Consider specifying a prior for the ``relative risk" (RR) of having a disease given that you carry one of two possible genetic types $A$ or $a$. Note $RR(A \mbox{ vs } a) := p(\mbox{disease} | A)/p(\mbox{disease} | a)$.
  
For most common diseases, such as heart disease, where the factors affecting risk
are many and complex, values of RR for a genetic factor are typically small (close to 1): for example, values $>1.2$ are generally considered uncommon. Also,
the relative risk of $A$ vs $a$, is the inverse of the relative risk of $a$ vs $A$.  So, unless we have particular reason to distinguish between $A$ and $a$, our prior on 1/RR should be the same as our prior on RR. One possible choice would be to put a Normal prior on log(RR) $\sim N(0,\sigma^2)$, with $\sigma^2$ chosen
so that $\Pr({\rm RR} >1.2) \approx 0.01$ (or 0.05? or 0.001?). 
Note: the normal distribution is a little inflexible,
and in particular has tails that decay very quickly.
This limits its ability to capture particular prior beliefs.
One can get more flexibility by using a $t$ distribution, or (sometimes more conveniently, and almost equivalently) a finite mixture of normals with different variances. See Stephens and Balding (Nature Reviews Genetics, 2009) for further discussion.
 \end{enumerate}

\subsubsection{Two Rules of Thumb}
\begin{enumerate}
\item If putting a prior on a parameter that is really a ratio of two symmetric quantities (e.g. the RR above, or the ratio
of the expression level of gene $A$ to gene $B$, where no information is available a priori to distinguish $A$ and $B$) then take the log of the parameter, and put a symmetric (perhaps normal, with some variance?) prior centered on 0.
\item If putting a prior on a small positive quantity, about which uncertainty spans orders of magnitude, put a prior (perhaps uniform, on some range, say?) on the log of the quantity, and not on the quantity itself. Example: imagine placing a prior on the probability $p$ that an aircraft you are about to board will crash. You decide that, for sure $p$ is between $10^{-12}$ and $10^{-5}$, but much more you are not prepared to say. This leads you 
to consider putting a $U[10^{-12},10^{-5}]$ prior on $p$.  But I would argue that this is a mistake. For example, it implies (approximately) a $90\%$ prior probability that $p \in [10^{-6}, 10^{-5}]$, 
which is probably not what you intended. Consider instead a uniform prior on $\log_{10}(p)$ in $[-12,-5]$. Although one could argue that the hard end-points are unnatural,
this prior does at least capture the idea that we are {\it uncertain about the order of magnitude} of $p$.
\end{enumerate}

\subsubsection{Prior Elicitation}

In many cases a statistician may be in the position of attempting to obtain a prior that
captures {\it someone else's} subjective beliefs. Usually this someone else would be a subject matter expert, who is not a statistician. The process of obtaining priors from subject
matter experts is known as {\it Prior Elicitation}, and is a research topic of its own.
See for example the book "Uncertain Judgements: Eliciting Experts' Probabilities'', published by Wiley.



\subsection{Conjugate Priors}

A family $\cal F$ of prior distributions for $\theta$ is said to be
{\bf closed under sampling} from a model $f(x\mid \theta)$ if for
every prior distribution $f(\theta)\in {\cal F}$, the posterior
distribution
$$f(\theta\mid x)\propto f(\theta)f(x\mid\theta)$$
is also in $\cal F$.  (It follows, e.g. by iteration, that the
posterior from a random sample $x_1,\cdots,x_n$, all with density
$f(x\mid \theta)$ will be in $\cal F$.)

Recall for instance,  if $X$ is $N(\theta,\phi)$ given
the values of the parameters $\theta$ and $\phi$.  Suppose the
variance is known and take a prior for $\theta$ which is
$N(\theta_0,\phi_0)$.  We saw that the posterior for $\theta$ was also
Normal. That is, the normal distribution is the conjugate prior for the mean of a normal distribution (with known variance).

\underline{Definition:}  Suppose $x_1,\cdots,x_n$ is a random sample from a
  regular k-parameter exponential family,
$$f(x\mid \theta)=f(x)g(\theta)\exp\left( \sum_{i=1}^k c_i \phi_i
  (\theta) h_i (x) \right).$$
 Then the prior distribution
  for $\theta$ of the form
$$p(\theta\mid\tau)=[K(\tau)]^{-1} [g(\theta)]^{\tau_0} \exp\left(
  \sum_{i=1}^k c_i \tau_i \phi_i (\theta)  \right),$$
where $\tau$ is such that
$$
K(\tau) = \int_{\Theta} g(\theta)^{\tau_0} \exp \left(\sum_{i=1}^k c_i \tau_i \phi_i(\theta)\right) d\theta < \infty,
$$
is said to be a {\it conjugate prior}.  


Note that the conjugate prior is a distribution for $\theta$.  The parameters of the prior, $\tau$, are often referred to as {\it hyperparameters}.  We will prove below that these
conjugate priors are closed under sampling.


\underline{Examples:}
\begin{enumerate}
\item Bernoulli Likelihood.  Recall that $f(x)=1$, $g(\theta)=1-\theta$, $h(x)=x$, $\phi(\theta)=\log\left(\frac{\theta}{1-\theta} \right)$, $c=1$.   Thus,
$$p(\theta\mid \tau_0,\tau_1) \propto (1-\theta)^{\tau_0} \exp\left(
  \log(\frac{\theta}{1-\theta} ) \tau_1\right)$$
\vskip 2mm
$$=\frac{1}{K(\tau_0,\tau_1)} \theta^{\tau_1}
(1-\theta)^{\tau_0-\tau_1}$$
The density will have finite integral iff $\tau_1> 1$ and $\tau_0-\tau_1>1$, in which case this is the Beta$(\tau_1+1, \tau_0-\tau_1+1)$ distribution.
\item Poisson Likelihood.  Recall $f(x)=(x!)^{-1}$,
  $g(\theta)=\exp{-\theta}$, $h(x)=x$, $\phi(\theta)= \log \theta$, $c=1$.  The conjugate prior is 
$$p(\theta\mid \tau_0,\tau_1) \propto \exp(-\tau_0 \theta) \exp(\tau_1
\log(\theta))$$
$$=\frac{1}{K(\tau_0,\tau_1)} \theta^{\tau_1}
e^{-\tau_0 \theta}$$
This is the Gamma$(\tau_1+1,\tau_0)$ distribution, assuming that
$\tau_1+1>0$ and $\tau_0>0$.
\item Normal Likelihood.  For convenience, we write
  $\lambda=1/\sigma^2$ for the {\it precision} of the normal
  model, and $\mu$ for the mean:
$$p(x\mid \mu,\lambda)=(\frac{\lambda}{2\pi})^{1/2}
\exp\left(-\frac{\lambda}{2} (x-\mu)^2 \right)$$
\vskip 2mm
Note here that $f(x)=1/\sqrt{2\pi}$, $g(\mu,\lambda)=\sqrt{\lambda}\,
 \exp(- \lambda \mu^2/2)$, $h(x)=(x,x^2)$, $\phi(\mu,\lambda)=(\mu
\lambda,\lambda)$, $c_1=1$, $c_2=-1/2$. 
The conjugate prior is then
$$p(\mu,\lambda\mid \tau_0,\tau_1,\tau_2)\propto
\left(\sqrt{\lambda}\, \exp(-\lambda \mu^2/2)\right)^{\tau_0} \exp\left(
  \mu \lambda \tau_1 - \frac{1}{2} \lambda \tau_2 \right)$$
provided $\tau_0,\tau_1,\tau_2$ are chosen so that this has finite
integral.  
\vskip 4mm
Thus
$$p(\mu,\lambda\mid \tau_0,\tau_1,\tau_2)\propto \lambda^{\frac{\tau_0+1}{2}-1} \exp\left(-\frac{1}{2}(\tau_2 -
  \frac{\tau_1^2}{\tau_0})\lambda\right) \lambda^{1/2} \exp\left(-\frac{\lambda \tau_0}{2} (\mu - \frac{\tau_1}{\tau_0})^2 \right)$$
On inspection, this has the form of a Gamma$(\frac{1}{2}(\tau_0+1),\frac{1}{2}
(\tau_2-\frac{\tau_1^2}{\tau_0}))$ density for $\lambda$ and conditional
on this a $N(\frac{\tau_1}{\tau_0},\frac{1}{\lambda \tau_0})$ density
for $\mu$.  This is sometimes called a {\it normal-gamma} distribution for
 ($\mu,\lambda$).  The integral will be finite and hence the prior
sensible, provided $\tau_2>\frac{\tau_1^2}{\tau_0}$, $\tau_0>0$.
\end{enumerate}
\vskip 4mm
\underline{Proposition:}  Consider a regular $k$-parameter exponential
family model.  The family of conjugate priors defined above for the
model is closed under sampling.  In fact, writing
$p(\theta\mid\tau_0,\tau_1,\ldots,\tau_k)$ for the parameterized
conjugate prior:

\ni 1) ~~The posterior density for $\theta$ is
$$p(\theta\mid x_1,\ldots,x_n,\tau_0,\ldots,\tau_k)=p(\theta\mid \tau_0+n,\tau_1+\sum_j h_1(x_j),\ldots,\tau_k+\sum_j
h_k(x_j))$$
so we get the same parametric form as the prior, but with parameters
$\tau_0+n$, $\tau_1+\sum_j h_1(x_j)$, $\ldots$, $\tau_k+\sum_j h_k(x_j)$.

\ni 2) ~~The predictive density for future observables
  $y=(y_1,\cdots,y_m)$ is 
\begin{eqnarray*}
&&p(y\mid x_1,\cdots,x_n,\tau_0,\cdots,\tau_k)= p(y\mid \tau_0+n,\tau_1+\sum_j h_1(x_j),\ldots,\tau_k+\sum_j
h_k(x_j))\\
&=&\prod_{l=1}^m f(y_l) \frac{K( \tau_0+n+m,\tau_1+\sum h_1(x_j)+\sum
h_1(y_l), \ldots,\tau_k+\sum h_k(x_j)+\sum h_k(y_l)) }{K(\tau_0+n,\tau_1+\sum h_1(x_j),\ldots,\tau_k+\sum h_k(x_j))}.
\end{eqnarray*}

\vskip 4mm
\underline{Remarks:}
\vskip 4mm
\begin{enumerate}
\item Recall that the sufficient statistics for the model are 
$$\left[ n, \sum h_1(x_j),\ldots,\sum h_k(x_j)\right]=[t_0,\ldots,t_k]$$
\item The inference process takes a very simple form.  The effect of
  data $x_1,\cdots,x_n$ is that the labelling parameters of the posterior
  are changed from those of the prior, $(\tau_0,\ldots,\tau_k)$, simply by the the addition of the sufficient
  statistics to give parameters $t_0+\tau_0,t_1+\tau_1,\ldots,t_k+\tau_k$, for the posterior.
  The effect of $x_1,\ldots,x_n$ on the predictive distribution is
  similar.
\end{enumerate}
\vskip 4mm
\ni\underline{Proof of the Proposition:}
\vskip 4mm
By Bayes Theorem,
\begin{eqnarray*}
&&p(\theta\mid x_1,\ldots,x_n,\tau_0,\ldots,\tau_k) \propto p(x_1,\ldots,x_n \mid\theta)p(\theta\mid
\tau_0,\ldots,\tau_k)\\
&=&\prod_{i=1}^n f(x_i) [g(\theta)]^n \exp\left(\sum_{i=1}^k c_i \phi_i(\theta)
  (\sum_{j=1}^n h_i(x_j)) \right)
 [K(\tau)]^{-1} [g(\theta)]^{\tau_0} \exp\left(\sum_{i=1}^k c_i
  \phi_i(\theta) \tau_i \right)\\
&\propto& [g(\theta)]^{n+\tau_0} \exp\left( \sum_{i=1}^k c_i \phi_i(\theta) (\tau_i+\sum_{j=1}^k h_i(x_j)) \right).
\end{eqnarray*}
This is of the form
$p(\theta \mid \tau_0+n,\tau_1+\sum h_1(x_j),\ldots,\tau_k+\sum
h_k(x_j))$.
\vskip 4mm
For the second part,
\begin{eqnarray*}
&&p(y\mid x_1,\ldots,x_n,\tau_0,\ldots,\tau_k)=\int_\Theta
p(y\mid\theta)p(\theta\mid x)d\theta\\
&=&\prod_{l=1}^m f(y_l) \left( K(\tau_0+n,\tau_1+\sum
  h_1(x_j),\ldots,\tau_k+\sum h_k(x_j)) \right)^{-1}\\
&&\times \int_\Theta [g(\theta)]^{\tau_0+m+n} \exp\left( \sum_{i=1}^k c_i \phi_i(\theta) \big( \tau_i+\sum_{j=1}^n
h_i(x_j)+\sum_{l=1}^m h_i(y_l) \big) \right)d\theta,
\end{eqnarray*}
as required.
\vskip 6mm
The use of conjugate priors simplifies Bayesian calculations.
They should be thought of as convenient tools.  It may be that in a
particular problem, one's prior beliefs will be well-approximated by a
certain member of the appropriate family of conjugate priors.  In
fact, mixtures of conjugate priors also lead to a simple
analysis ({\it Exercise}).


\subsection{Objective Priors}

\subsubsection{What's in a name?}

Many different terms are used to refer to 
priors that are chosen for reasons other than
strictly representing your (or someone else's) prior uncertainty. Objective is one name, chosen
no-doubt because it sounds, well, objective. 
Other related terms used include ``non-informative", and ``default" priors. The term ``non-informative prior" has been used for a while, but I am not keen on it because in general I think there is no such thing as a ``non-informative" prior - the result can always change with a different prior, so in that sense the prior is always ``informative".  
Objective is arguably a better term, but it could
be accused of hiding the fact that whether or not a particular objective prior is appropriate for a given
application may be subjective! For example, suppose you derive a prior on the basis of its being invariant to
certain transformations of the parameter space and/or data. It may be a subjective (or context-dependent) question whether this invariance is desirable in a given context!
I like ``default" best, as I think it captures what we are trying to achieve. (Imagine distributing a software package for Bayesian analysis of linear regression; what defaults would you have for the prior on the regression parameters? You could
give no defaults, but then no-one would use your package....!) But
Objective is perhaps the one most modern researchers working in this area use (there are
conferences now on ``Objective Bayes"). 

Some other terms that you may come across are ``Jeffrey's Priors" and ``Reference Priors". These are {\it particular} approaches to obtaining Objective priors. Jeffrey's priors are due to 
Harold Jeffreys, and we will look at them in more detail. The term Reference Priors is due to Jos\'e Bernardo, and have also been worked on by others, including Jim Berger.

People also use terms like ``vague prior", ``diffuse prior", or ``flat prior". I tend to think of these as
descriptive terms rather than technical terms. Usually it suggests to me that the authors think
that that results will be robust to this choice (e.g. because the data will be highly informative about this parameter), and so did not bother to think too hard about exactly what prior should be chosen.
(So if I see those terms, and think that the results
will not be robust, then I worry.)

Finally, another related term you need to be familiar with is ``improper prior". This is used to refer to a function $p(\theta)$ that does not integrate to a finite quantity (so it is not a distribution, never mind a prior distribution), but which is used in place of the prior distribution in Bayes Theorem, to compute the posterior by
$p(\theta | D) \propto p(D|\theta) p(\theta)$. 
The idea is that even if $p(\theta)$ does not integrate to a finite quantity,
if $\int p(D|\theta) p(\theta) d\theta$ is finite then
the above produces a ``proper" posterior. 
\begin{enumerate}
\item For example, suppose $x | \theta \sim N(\theta,1)$. Assume an ``improper prior" on $\theta$ with $p(\theta) \propto 1$. This is improper because $\int p(\theta) d\theta$ is infinite, and so $p(\theta)$ is not proportional to a density. However, if we just ``plug in" this prior to Bayes Theorem we get
$p(\theta | x) \propto \exp(-(\theta-x)^2/2)$, which implies that $\theta | x \sim N(x,1)$. That is we get a ``proper" posterior.
\item Improper priors often result from attempts to define Objective priors. See, for example, below.
\item  Often, Bayesian analysis with an improper prior can lead to
  procedures with attractive frequentist properties.  In fact, many
  classical procedures and estimators correspond to Bayesian analysis
  with particular improper priors.
\item Improper priors can never be ``subjective" priors, but they can be ``semi-subjective" (e.g. Savage's potato). That is, the posterior that arises from an improper prior may be viewed as a good approximation of a
  ``proper'' analyses.
\end{enumerate}

So is it OK to use an improper prior? A strict subjectivist  might argue that it is never OK, because it cannot represent your prior uncertainty.
Certainly improper priors {\it can} lead to paradoxes and inconsistencies, which proper priors never do
(because they obey the axioms of probability).
 However, in many cases an improper prior produces a posterior that is a good approximation to the posterior that one would have got with a carefully-constructed
subjective proper prior. In these cases (as in the semi-subjective prior above) I would argue it is OK. However, it is wise to be careful before using an improper prior. {\it In particular you should check that your posterior distribution is proper, as this is not always the case!}


\ni\underline{Example:} The problem with improper priors for mixtures

Consider an observation $x$ from an equal mixture of two normal distributions, one with unknown mean: $x \sim 0.9 N(\mu,1) + 0.1 N(0,\sigma^2 = 100)$. (Although this is a toy example, a possible motivation for something along these lines would be to do ``robust" estimation of $\mu$, where the second component is to allow for 10\% of the observations to be ``outliers".)
Show that attempting to use an improper prior for $\mu$ does not work (i.e. it leads to an improper posterior).


%\ni\underline{Example 3:}
%Consider the regression model:
%$$y= x \beta+\epsilon.$$
%with $\epsilon \sim N(0,\sigma^2)$.
%Consider comparing the model $H_0:\beta=0$
%with $H_1: \beta \neq 0$. In the latter case
%we need to specify some prior on 
%It may seem tempting to use the prior

\subsubsection{There is no such thing as an non-informative prior!}

Before we look at some ways to define Objective/non-informative priors, we should make it clear that in realistic settings there are almost always subjective issues that need addressing.

Consider, for example, the simplest case of
a  finite parameter space $\Theta$, with $|\Theta|=n$.  The
``obvious" Objective prior may seem to be to place mass $n^{-1}$ at each possible
parameter value. However, when the parameter space has ``structure" (e.g. some values of the parameter are more similar to one another than others) then this can be a lot less objective,
and a lot less sensible, than it might seem.
For example, consider an undirected graph with $V$ vertices. Then the number of possible edges is $E= {V\choose2}$, and the number of possible graphs is $G=2^E$. Now consider putting a prior on graphs. The uniform prior on graphs would put mass $1/2^E$ on each possible graph. This is the same as each edge being present, independently, with probability 0.5. So the number of actual edges present is, a priori, Binomial$(E, 0.5)$. If $E$ is large, as it usually is in this setting, then this prior is very concentrated near $E/2$. So this prior says that, with high probability, close to a half of the possible edges will be present - not very Objective at all (and in many settings not very sensible either - for example, in many applications one expects to see ``sparse" graphs).

An alternative, and (in my subjective opinion, generally more sensible) prior here is to assume that each possible edge is present, independently, with probability $p$, and then to place a Be$(1,1)$ or Be$(0.5,0.5)$ prior on $p$.  Alternatively, and equivalently,
assume that the number of edges is uniform on $0$ to $E$, and that, conditional on the number of edges, all graphs are equally likely. Note that in some contexts this prior will also not be appropriate. For example, in practice, one often expects some nodes of the graph to be connected to a lot of neighbors, while others will be connected to none. So maybe edge $i,j$ should be present with probability $p_{ij}$ given by $\log[p_{ij}/(1-p_{ij})] = \mu+\alpha_i + \alpha_j$ where
$\alpha_i$ and $\mu$ now need some prior specifying.....

This example shows how when attempting to be ``uninformative" for one feature (the actual graph)
one can easily, unintentionally, be extremely informative about another feature (the number of edges in the graph). Indeed, in general, in complex problems, the first aim of prior specification could be viewed as trying to find a prior that does not have unintended, undesirable, implications.

\subsubsection{Lindley's paradox}

Lindley (1957) gave an example of what he referred to 
as ``A Statistical Paradox". Here is his first sentence:

\begin{quote}
An example is produced to show that, if $H$ is a simple hypothesis and $x$ the result of an
experiment, the following two phenomena can occur simultaneously:
\begin{enumerate}[(i)]
\item a significance test for $H$ reveals that $x$ is significant at, say, the 5\% level;
\item the posterior probability of $H$, given $x$, is, for quite small prior probabilities of $H$, as high as 95 \%.
\end{enumerate}
\end{quote}

This results was followed by a discussion by Bartlett (1957), who corrects
an error in one of Lindley's expressions, 
and a similar result was apparently also known by Jeffrey's (e.g. Jeffreys 1961), and so the paradox 
is sometimes called the Jeffrey's paradox, or the Lindley-Jeffrey's paradox,
or the Bartlett paradox, or...

Lindley's example is basically the following. Suppose $x_1,\dots,x_n \sim N(\mu,1)$ and consider testing $\mu=0$. Then a sufficient statistic
is the sample mean $\bar{x}$, which has distribution $N(\mu,1/n)$. 
So we'd get a $p$ value of $0.05$ if $\bar{x} = 1.96/\sqrt(n)$ say.
Lindley also computes the posterior probability on $H_0$, for this value of $\bar{x}$, and a uniform prior for $\mu$ on some interval $I$. Of course this posterior probability depends on $n$, but as $n \rightarrow \infty$
he shows that the posterior probability on $H_0$ tends to 1.
Based on this, you can obviously have some value of $n$ for which the
$p$ value is 0.05 and the posterior probability on $H_0$ is 0.95 (or even higher) which supports his original claim.

Whether this is really a ``paradox" is a matter of opinion. It certainly says
that a $p$ value of 0.05 can correspond to very
weak evidence against $H_0$, but at this point this is not news to us!

\subsection{Bartlett's paradox, and problems with non-informative priors and Model choice}

In the discussion of Lindley's paradox by Bartlett (1957), he points out that Lindley missed a term $1/|I|$ in his prior specification where $|I|$ denotes
the length of the interval $I$. (Uniform on $I$ has density $1/|I|$).
Since this missing term might be considered constant in $n$, it does not affect Lindley's limiting argument on $n$. However, Bartlett then goes on
to point out another issue, which is that you might like to let $I$ be big,
 and that in the limit $|I| \rightarrow \infty$ then the posterior probability on $H_0$ tends to 1 (for any fixed data and fixed $n$), which he regards
 as a ``silly answer".

This result, which I have seen referred to as ``Bartlett's paradox" (although
Bartlett does not refer to this result as a paradox, and the name Bartlett's paradox is also sometimes used to refer to as Lindley's paradox!) effectively
says that very diffuse priors can give silly answers when doing model comparisons. That is, there is a subtle trap ready for the unwary who
are ready to simply use a ``convenient" prior without
thinking about it too much.
 In other words, for model comparison, it can
be crucial to give serious consideration to
suitable prior specification. 
 
To illustrate the idea, consider
$x_1,\dots,x_n \sim N(\mu,1)$ and consider comparing the hypothesis $H_0: \mu=0$ with the alternative $H_1: \mu \neq 0$. 
To do this is a Bayesian framework we would
place priors on $H_0$ and $H_1$ ($\pi_0$ and $\pi_1$ say), and compute the Bayes Factor for $H_0$ vs $H_1$,
BF= $p(x | H_1)/p(x|H_0)$, from which we could
then compute the posterior probabilities $p(H_0|x)$ and $p(H_1 | x)$. 

However, as we state it above, the alternative $H_1$ is not sufficiently  detailed to compute $p(x | H_1)$. Thus, to compare these hypotheses in a Bayesian framework, one must be more specific
about $H_1$. In particular we need to specify a prior distribution $p_1(\mu)$ for $\mu$ under $H_1$.

Putting this another way, distinguishing between
$\mu=0$ and $\mu \neq 0$ is a question about $\mu$, and so to address it from a Bayesian perspective we have to specify a prior distribution for $\mu$. However, simply saying that with probability $\pi_0$ $\mu$ is equal to 0, otherwise
it is not equal to 0, is incomplete as a specification
of a prior distribution! To complete it we have to say how it $\mu$ is distributed when it is not equal to 0. That is, we have to specify a prior distribution $p_1(\mu)$ for $\mu$ under $H_1$.

Having specified $p_1(\mu)$, we have the BF
$$\text{BF} = \frac{p(x | H_1)}{p(x|H_0)} = \frac{\int p(x | \mu) p_1(\mu) \, d\mu}{p(x | \mu=0)}.$$
Suppose now we choose the prior $p_1(\mu)$ to be a normal distribution with mean 0, and variance 
$\sigma_\mu^2$. It might be tempting to choose $\sigma_\mu^2$ to be ``big", thinking of this as ``non-informative". However, this would be a big mistake. Specifically, under this prior, as $\sigma_\mu^2 \rightarrow \infty$, the Bayes Factor tends to 0 (i.e.~provides infinite support for $H_0$, vs $H_1$).

To prove this, we can use the following trick
that can be useful for computing integrals of this sort. Note that, rearranging Bayes Theorem,
we get
$$\int p(x | \theta) p(\theta) \,d\theta = p(x | \theta) p(\theta) / p(\theta | x).$$
Note that, despite appearances, the right hand side does not actually depend on $\theta$ (since the LHS does not!) So if we know the posterior distribution, the prior distribution, and the likelihood, we also implicitly know the integral.

In the example above, $\theta$ is $\mu$.
Under $H_1$, in the limit $\sigma_\mu^2 \rightarrow \infty$, the terms $p(x | \mu)$ and $p(\mu | x)$ tend to finite limits, and $p_1(\mu | \sigma_\mu^2) \rightarrow 0$, so the integral tends to 0.

To summarize: when 
you are computing Bayes Factors to do model comparison
you have to be careful when 
using improper priors, or even ``very flat" proper priors, because
the BF can depend very sensitively to how flat the prior is. 
This is important:
if you didn't know about this paradox you might go ahead and use a diffuse prior
to compute Bayes Factors and not realize that your results are very sensitive to how flat.

The need for care in prior specification for BFs applies particularly to parameters that require a prior distribution in only one of the two models.
Often times there are additional (nuisance) parameters that 
occur in both models, and require a prior distribution in both models.
For these nuisance parameters you can often get sensible
BFs using very flat or improper priors on the nuisance parameters,
provided the same prior is used for both models.
 See \cite{servin.stephens.07} for example.

%So to summarize this:
%- if you are doing inference for a normal mean parameter $\theta$, having observed
%data $x| \theta \sim N(\theta,1)$, and use a prior distribution $\theta \sim N(0,\sigma_0^2)$,
%then the posterior distribution on $\theta$ tends to a sensible limit as $\sigma_0 \rightarrow \infty$.
%
%- However, if you consider {\it testing} $H_0:\theta=0$ in the above situation, vs $H_1:  \theta \sim N(0,\sigma_0^2)$
%by computing the BF for $H_1$ vs $H_0$
%then the BF does not tend to a sensible limit as $\sigma_0^2 \rightarrow infty$. (The BF tends to 0, whatever the value
%of $x$, so in that limit one would always not reject $H_0$, whatever the data.)
%
%\subsubsection{Invariance: Priors for Location Parameters}

%Suppose the parameter space $\Theta$ and the range $\cal{X}$ of the r.v.'s
%are subsets of Euclidean Space.  If the density of $x$ given
%$\theta$ is of the form $f(x-\theta)$ (so that it depends only on the
%value of $x-\theta$) the density is said to be a {\it location density}, and
%$\theta$ is called a location parameter.
%\vskip 4mm

%Now suppose we require that inference for $\theta$ transform in the following way to shifts in $y$: if we add $c$ to every observed $y$ then we want
%our posterior for $\theta$ to also shift by $c$.

%
%Consider observing
%$y=x+c$ for some fixed $c$.  If $\eta\equiv\theta+c$, it is clear that $y$
%has density $f(y-\eta)$.  The $(x,\theta)$ and $(y,\eta)$ problems are
% thus similar in structure.  If $\Theta={\bf R}^p$ for some p, they also have the same
%parameter space.  In this case, it is natural to insist that a
%noninformative prior should be the same in each problem.
%\vskip 4mm
%If $\pi$ and $\pi^*$ denote noninformative priors for $(x,\theta)$ and
%$(y,\eta)$ respectively, we would then require
%$$
%\int_A \pi(\theta)d\theta = \int_A \pi^* (\eta)d\eta
%=\int_{A-c} \pi(\theta)d\theta=\int_A \pi(\theta-c)d\theta
%$$
%where $A-c=\{z-c:z \in A\}$, since $\eta\equiv\theta+c$.  For this to
%hold for all measureable $A$, we must have $$
%\pi(\theta)=\pi(\theta-c)
%$$
%for all $\theta$.  In particular, we must have $$
%\pi(c)=\pi(0).
%$$  For
%this to hold for all $c$, $\pi$ must be constant, giving the improper
%noninformative prior for a location parameter as $\pi(\theta)=1$.
%(Note that any constant will do; the prior is proportional to Lesbesque
%measure.)
%\vskip 4mm
%\subsubsection{Scale Parameters}

%A (one-dimensional) {\it scale density} is a density of the form
%$\sigma^{-1}f(x/\sigma)$ where $\sigma > 0$.  The parameter $\sigma$
%is called a {\it scale parameter}.
%\vskip 4mm
%For a noninformative prior, consider $y=cx$ where $c>0$.  If
%$\eta=c\sigma$, it follows that y has density $\eta^{-1}f(y/\eta)$.
%The $(x,\sigma)$ and $(y,\eta)$ problems now have the same structure.
%It can be argued that they should then have the same prior.
%\vskip 4mm
%Arguing analogously as for location parameters, this requires
%$$\int_A \pi(\sigma)d\sigma=\int_{c^{-1}A} \pi(\sigma)d\sigma
%=\int_A \pi(c^{-1}\sigma)c^{-1} d\sigma,
%$$
%where $c^{-1} A=\{c^{-1} z ; z\in A\}$.  This requires
%$$\pi(\sigma)=c^{-1}\pi(c^{-1} \sigma)$$ for all $\sigma$.  In
%particular, $$\pi(c)=c^{-1} \pi(1)$$.  Since $c$ is arbitrary we must have
%$\pi(\sigma)\propto \sigma^{-1}$.  This noninformative prior for a
%scale parameter is improper.   Without loss of generality, write
%$\pi(\sigma)=\sigma^{-1}$.
%\vskip 4mm
%The noninformative priors for location or scale parameters just
%derived are improper.  The argument that ``since the $(x,\theta)$ and
%$(y,\eta)$ problems are ``the same'', they must have the same prior''
%does not apply if the prior is improper.  All that is required is that
%$\pi$ and $\pi^*$ agree up to a multiplicative constant.  In fact
%there will usually then be a wide class of noninformative priors.
%\vskip 4mm
%In the location example, with $\Theta\in {\bf R}$, if
%$\pi(\theta)=e^{z\theta}$ for fixed z, then
%$$P_\pi (\theta\in A)=h(c) P_\pi (\eta\in A).$$
%\vskip 4mm
%For statistical problems with a certain group invariance structure, it
%turns out that there is a natural choice, namely the right invariant
%Haar measure.  The location and scale invariant priors
%($\pi(\theta)=1$, $\pi(\sigma)=\sigma^{-1}$) do correspond to this choice.

\subsubsection{Limiting forms of conjugate priors as ``non-informative" priors}

It often turns out that the hyper-parameters of a conjugate prior have an effect on the posterior
that is reasonable straightforward to interpret. In these cases it is often tempting (and sometimes
reasonably sensible) to think of ``limiting" forms for these priors as being ``minimally informative". 

\ni\underline{Example:} Limiting Beta prior for Binomial proportion
\vskip 2mm
Consider observing the number of successes ($n_s$) and failures ($n_f$)
in a series of $n$ Bernoulli($p$) trials. 
The conjugate prior is $p \sim Beta(\alpha, \beta)$
and corresponding posterior is $p | n_s,n_f \sim Beta(n_s+ \alpha, n_f + \beta)$, with posterior mean $(n_s+\alpha)/(n+\alpha+\beta)$. Note that,
informally, $\alpha$ and $\beta$ can be interpreted as an ``effective number of successes and failures" that the prior encapsulates, because $\alpha$ is added to $n_s$ and $\beta$ is added to $n_f$. Certainly, if $\alpha$ and $\beta$ are big, then the prior will outweigh the data, and the larger $\alpha$ and $\beta$ are the more ``informative" the prior is. Based on these kinds of  argument one might argue that taking $\alpha$ and $\beta$ to be as small as possible will make the prior ``minimally" informative. This could lead you to take the limit $\alpha, \beta \rightarrow 0$. The resulting posterior is the same as you would have gotten if you used the improper prior $$\pi(p) \propto p^{-1} (1-p)^{-1}$$ which is sometimes referred to as Haldane's prior. Note: in this case the posterior is proper provided both $n_s$ and $n_f$ are $>0$; otherwise it is improper!

\ni\underline{Example:} Limiting conjugate prior for Normal mean and Variance
\vskip 2mm

Suppose we observe data $x=x_1,\dots,x_n$ are i.i.d.~ $\sim N(\mu,1/\tau)$ where $\tau$ here indicates the inverse of the variance, also known as the ``precision". [It turns out that algebra tends to be easier if we work with $\tau=1/\sigma^2$, so this is common in this context.]

The joint conjugate prior for $(\mu,\tau)$ is 
the ``Normal-gamma" with hyperparameters $(\mu_0,n_0, m_0,l_0)$:
\begin{align}
\tau  & \sim \Gamma(m_0/2,l_0/2) \\
\mu | \tau  & \sim N\left(\mu_0, (1/n_0) (1/\tau)\right).
\end{align}

With this prior, the posterior can be derived as
also being Normal-gamma:
\begin{align}
\tau | x & \sim \Gamma(m_1/2,l_1/2) \\
\mu | \tau, x & \sim N\left(\mu_1, (1/n_1) (1/\tau) \right).
\end{align}
where 
\begin{align}
m_1 & = m_0+n \\
l_1 & = l_0+ \sum_{i=1}^n (x_i - \mu_1)^2 \\
\mu_1 & = \lambda \bar{x} + (1-\lambda) \mu_0 \\
n_1 & = n_0+n
\end{align}
where $\lambda = n/(n+n_0)$.

Note that, intuitively,  $n_0,m_0$ can be thought of as representing an ``effective number of observations" informing the prior. 

Question: what happens to the posterior in the limit $n_0, l_0,m_0 \rightarrow 0$?
Is there an ``improper prior" that
gives the same posterior as this limiting posterior?
What is it? (Ans: $p(\mu,\tau) \propto \tau^{-1/2}$,
or equivalently, $p(\mu,\sigma) \propto 1/\sigma^2$. This is also the multivariate Jeffreys prior, which Jeffrey's ultimately rejected (see later).

\ni\underline{A  paradox:} Limiting conjugate prior for Normal mean and Variance
\vskip 2mm

The above example leads to an interesting paradox. Consider the case $n=1$ and $n_0 \rightarrow \infty$. Then the limiting posterior distribution for $\tau | x$ is  $\Gamma( (m_0+1)/2,l_0/2)$. Note that this is different from the
prior, but does not depend on $x_1$! So we can simply ``imagine" collecting $x_1$, update
our prior, and get information on $\tau$.
Note that this paradox occurs only in the limit
(or equivalently, only with the use of an
improper prior). Paradoxes like this
never occur with proper priors. For this reason
some people argue that you should never use improper priors. 

Note, however, that {\it in practice}, for realistic $n >>1$ the limiting posterior will
be a very good approximation to the posterior
you would get with a proper prior. So I don't think
this example completely rules out the use of such a prior in practice, as a way to get a reasonable approximation to your posterior. Rather, it illustrates the {\it conceptual} or {\it foundational} difficulty of defining
suitable universal ``non-informative" priors. 




\subsubsection{Jeffreys Priors}

The above arguments are rather {\it ad hoc} (if, possibly, intuitively appealing)
ways to come up with non-informative priors. 
In contrast Jeffrey's came up with a much more formal way to derive potentially ``non-informative" priors.

Jeffreys wanted to come up with a ``rule" for obtaining an objective prior that had the following property: if you applied the rule to $\theta$
you would get the same prior for any (monotone differentiable) 1-1 transformation of $\theta$, $h(\theta)$ say, as if you applied the rule to $h(\theta)$. In this sense the prior (or the rule) is ``transformation invariant".

Note that the rule ``use a uniform prior" doesn't work in general. For example, consider $x\sim$ Binomial$(n,p)$.  Now suppose we apply this rule to $p$, so $p\sim$ U[0,1]. And if we apply the rule to $\sqrt{p}$, we get $\sqrt(p) \sim $U[0,1]. But these are different  prior assumptions.




\ni\underline{Definition:}  Recall that for a model with parameter space
$\Theta \subseteq \bf{R}$, the Fisher information is
$$I(\theta)=E_{\theta} \left(\frac{d\log(f(x\mid \theta))}{d\theta}
\right)^2$$
where $f(x\mid \theta)$ is the sampling distribution and the
expectation is taken over $f(x\mid \theta)$.  Under regularity
conditions,
$$I(\theta)=-E_{\theta} \left(\frac{d^2 \log(f(x\mid \theta))}{d\theta^2}
\right).$$
In such a setting, the {\it Jeffreys Prior} for $\theta$ is defined by
$\pi(\theta)\propto I(\theta)^{1/2}$, to be proportional to the square
root of the Fisher Information at $\theta$.  Note that in general the
Jeffreys prior may be improper.
\vskip 4mm
Note that by the chain rule,
$$I(\theta)=I(h(\theta)) \Big( \frac{dh}{d\theta} \Big)^2.$$
If $\theta$ has the Jeffreys prior and $h$ is a monotone differentiable
function of $\theta$, the prior induced on $h(\theta)$ by the Jeffreys
prior on $\theta$ is
$$\pi(h(\theta))=\pi(\theta) \big| \frac{dh}{d\theta} \big|^{-1}
\propto I(\theta)^{1/2} \big| \frac{dh}{d\theta}
\big|^{-1}=I(h(\theta))^{1/2}.$$
Thus the Jeffreys priors are invariant under reparameterization.

Recall the interpretation of $I(\theta)$ as the ability of the data to
distinguish between $\theta$ and $\theta+d\theta$.  If the prior
favors values of $\theta$ for which $I(\theta)$ is large, the effect
is to minimize the effect of the prior distribution relative to the
information in the data. In this sense you can 
think of the prior as attempting to be ``uninformative'' about
$\theta$.

\vskip 6mm
\ni\underline{\bf Example:}
\vskip 4mm
Suppose $x\sim$ Binomial$(n,p)$.  Then:
$$f(x\mid \theta)={n \choose x}p^x (1-p)^{n-x}$$
so that
$$\frac{d^2 \log(f(x\mid p))}{dp^2}
=-\frac{x}{p^2} - \frac{(n-x)}{(1-p)^2}.$$
Thus
\begin{eqnarray*}
I(p)&=&E\left( \frac{x}{p^2} + \frac{n-x}{(1-p)^2} \right)\\
&=&\left( \frac{n}{p} + \frac{n}{1-p} \right) = \frac{n}{p(1-p)}.
\end{eqnarray*}
Thus the Jeffreys prior for $p$ is
$$\pi(p)\propto [p(1-p)]^{-1/2},
$$
which is the Beta$(\frac{1}{2},\frac{1}{2})$ density (and hence proper).
\vskip 8mm




\subsubsection{Multivariate Jeffrey's Priors}

For $\Theta \subseteq {\bf R}^k$, under suitable regularity conditions
the Fisher Information matrix has
$(i,j)$th element
$$I_{ij}(\theta)=-E_{\theta}\left( \frac{\partial^2}{\partial\theta_i \partial\theta_j}
  \log f(x\mid \theta) \right).$$

In this
multidimensional case, the Jeffreys prior is defined by
$$\pi(\theta)\propto [\mbox{det}(I(\theta))]^{1/2}.$$
It is still invariant under reparameterization.  
However, it leads to priors that are considered undesirable by some
(including Jeffreys himself).

\ni\underline{Example (Bernardo and Smith, p361)}

Consider $x\sim N(\mu, \sigma^2)$ with $\theta=(\mu, \sigma)$.

Then,
\begin{eqnarray*}
I(\theta)&=&-E_\theta \left(\begin{array}{ll}
 \frac{\partial^2}{\partial\mu^2}(-\log\sigma -\frac{(x-\mu)^2}{2\sigma^2})
 &\frac{\partial^2}{\partial\mu\partial\sigma}(-\log\sigma -\frac{(x-\mu)^2}{2\sigma^2})\\
\frac{\partial^2}{\partial\mu\partial\sigma}(-\log\sigma -\frac{(x-\mu)^2}{2\sigma^2})
&\frac{\partial^2}{\partial\sigma^2}(-\log\sigma -\frac{(x-\mu)^2}{2\sigma^2})
\end{array}\right)\\
&=& -E_\theta \left(\begin{array}{cl}
-1/\sigma^2 & 2(x-\mu)/\sigma^3\\
2(x-\mu)/\sigma^3 & 1/\sigma^2 - 3(x-\mu)^2/\sigma^4)
\end{array}\right)\\
&=&\left(\begin{array}{cc}
\frac{1}{\sigma^2}&0\\
0&\frac{2}{\sigma^2}
\end{array}\right).
\end{eqnarray*}
The Jeffreys prior is thus
$$
\pi(\mu,\sigma)\propto
\left(\frac{1}{\sigma^2}\frac{2}{\sigma^2}\right)^{1/2} \propto
\frac{1}{\sigma^2}.
$$

Using this prior to analyse data $x_1,\dots,x_n$ leads to a posterior in which $\sum_{i=1}^n (x_i - \bar{x})^2 / \sigma^2$ is $\chi_n^2$. Compare this to an analysis of data with {\it known} mean $0$,
and unknown variance; in this case the posterior is such that $\sum_{i=1}^n x_i^2/\sigma^2$ is 
$\chi_n^2$. Since the degrees of freedom of these $\chi^2$ variables are the same, it seems that not knowing the mean does not lead to any loss of information
in the posterior, which is widely recognized as unacceptable. 

This kind of problem led Jeffreys to the ad hoc recommendation
of treating location and scale parameters independently, applying his rule to each group
separately, and multiplying the results together.
For the  $N(\mu,\sigma^2$) density this leads to $$\pi(\mu, \sigma) = \frac{1}{\sigma}$$
instead of $\sigma^{-2}$. This leads to a posterior in which $\sum_{i=1}^n (x_i - \bar{x})^2 / \sigma^2$ is $\chi_{n-1}^2$, and we have ``lost a degree of freedom" by not knowing the mean, which accords
better with intuition. 

\subsubsection{Marginalisation Paradoxes}

Consider example 5.26 in Bernardo and Smith, p362.
This example assumes $x_1,\dots,x_n$ to be a random sample from $N(\mu, \sigma^2)$.
The standard noninformative prior for this problem is $\pi(\mu,\sigma) = 1/\sigma$. Although this
gives adequate results if one wants to make inferences about either $\mu$ or $\sigma$, it is quite
unsatisfactory if one wants to make inferences about $\psi = \mu/\sigma$.
Specifically, it turns out that:
\begin{itemize}
\item the posterior for $\psi$ depends on the data $x$ only through
the statistics $t = \sum_i x_i / \sum_i x_i^2$.
\item the sampling distribution of $t$ depends on $\mu,\sigma$ only
through $\psi$.
\item the posterior for $\psi$ is {\it not} proportional to any prior on $\psi$ times
this sampling distribution.
\end{itemize}
That is, in some sense the posterior
that one obtains for $\psi$ using this prior does not correspond to what one would get
with any given prior on $\psi$ (Stone and Dawid, 1972)!

Bernardo and Smith note that this type of {\it marginalisation paradox} appears in many multivariate
problems, and makes it ``difficult to believe that, for any given model, a {\it single} prior may be usefully
regarded as ``universally" non-informative".

\subsubsection{Maximum Entropy Priors}

The approach of maximum entropy priors
was pioneered by ET Jaynes. He assumed that there existed a certain limited
amount of information (e.g.~the prior mean and variance), 
and then tried to determine a prior that reflected this initial information,
and nothing else (or with minimal additional information). 
The basic idea is to maximise the uncertainty of the prior (as measured by Entropy, defined
below), subject to constraints determined by the initial information. It is a nice
idea that works well for discrete parameters, but not continuous ones.

Suppose that $\Theta$ is discrete.  If $\pi$ is a probability mass
function on $\Theta$  then the
{\it entropy} of $\pi$, denoted by ${\cal{E}}(\pi)$, is defined by
$${\cal{E}} (\pi)= - \sum_{\Theta} \pi(\theta_i) \log(\pi(\theta_i))$$
(with x log x $\equiv$ 0 for x = 0).

Entropy is a natural
measure of the amount of uncertainty in an observation from the
distribution.  For example, if $\Theta$ is finite with $|\Theta |=n$,
the entropy is largest for the uniform distribution on $\Theta$ and
smallest whenever $\pi(\theta_i)=1$ for some $\theta_i$.
\vskip 4mm
Now suppose partial prior information is available so that for $m$
functions $g_1,\ldots,g_m$, we must have
\begin{equation}
E_{\pi} (g_k(\theta))\equiv \sum_{i} \pi(\theta_i) g_k(\theta_i)=\mu_k
\end{equation}
under the prior $\pi$.  In choosing amongst the priors which satisfy
these constraints, it may be natural (in terms of representing
ignorance) to choose the distribution with the maximal entropy.  It
can be shown that amongst all probability distributions which satisfy
the constraints (1), the distribution with maximal entropy is
$$\tilde{\pi} (\theta_i)={\exp \left( \sum_{k=1}^m \lambda_k
    g_k(\theta_i) \right) \over \sum_i \exp \left( \sum_{k=1}^m \lambda_k
    g_k(\theta_i) \right)}$$
where the $\lambda_i$ are constants whose values will be determined by
the constraints.
\vskip 4mm
\ni\underline{\bf Example:}  Suppose $\Theta=\{0,1,2,\ldots\}$ and the
prior mean of $\theta$ is thought to be 5.  This is a constraint with
$m$=1, $g_1(\theta)=\theta,\mu_1=5$.  Then the maximum entropy prior is
$$\tilde{\pi} (\theta)={e^{\lambda_1 \theta} \over \sum_{j=0}^\infty
  e^{\lambda_1 j}} = (e^{\lambda_1})^\theta (1-e^{\lambda_1}),$$
a geometric distribution, and we need $e^{\lambda_1}=\frac{1}{6}$ to
ensure it has mean 5.
\vskip 6mm
Now consider the case in which $\Theta$ is continuous.  The use of
maximum entropy is more complicated, mostly because there is no
longer a completely natural definition of entropy.
\vskip 4mm
Suppose $\pi(\theta)$ is a density.  The entropy of $\pi$ relative to
a particular ``reference'' distribution with density $\pi_0$ is
$${\cal{E}}(\pi)=-E_{\pi} \left( \log\frac{\pi(\theta)}{\pi_0 (\theta)}
\right) = -\int_\Theta \pi(\theta)\left( \log\frac{\pi(\theta)}{\pi_0
    (\theta)} \right) d\theta.$$
(The definition of entropy in the discrete setting above
coincides with this definition if $\pi_0 (\theta)$ is taken to be
uniform.)
\vskip 4mm
In the discrete setting the natural reference measure $\pi_0$
is uniform.  For continuous $\Theta$, the choice of reference measure
is less clear.  Note that in general, a prior which is chosen to have
maximum entropy with respect to $\pi_0$ subject to certain
constraints will depend on the choice of reference measure 
$\pi_0$.
\vskip 4mm
One approach is to use as $\pi_0$ the natural ``invariant''
noninformative prior for the problem.  In the presence of partial
information of the form
$$\int_{\Theta} g_k(\theta) \pi(\theta) d\theta = \mu_k, \quad k=1,
\ldots, m,$$
the (proper) prior density (satisfying these restrictions) which
maximizes entropy relative to $\pi_0$ is given (when it
exists) by
$$\tilde{\pi} (\theta)={\pi_0 (\theta) \exp \left( \sum_{k=1}^m
    \lambda_k g_k(\theta) \right) \over \int_{\Theta} \pi_0 (\theta)
  \exp \left( \sum_{k=1}^m \lambda_k g_k(\theta) \right) d\theta}$$
where the $\lambda_k$ are constants to be determined by the
constraints.
\vskip 4mm
\ni\underline{\bf Example 1:}~~  Suppose $\Theta={\bf R}$ and that $\theta$
is a location parameter.  The natural noninformative prior is then
$\pi_0 (\theta)=1$.  If the prior mean and variance of $\theta$ are
fixed to be $\mu$ and $\sigma^2$ respectively, we have
$g_1(\theta)=\theta,\mu_1=\mu$; $g_2(\theta)=(\theta-\mu)^2,
\mu_2=\sigma^2$.  The corresponding
maximal entropy prior is
\begin{eqnarray*}
\tilde{\pi} (\theta)&=&{\exp(\lambda_1 \theta + \lambda_2
  (\theta-\mu)^2) \over \int_\Theta \exp(\lambda_1 \theta + \lambda_2
  (\theta-\mu)^2) d\theta}\\
&\propto&\ exp(\lambda_1 \theta + \lambda_2 \theta^2) \\
&\propto& \exp(\lambda_2 (\theta-\alpha)^2)
\end{eqnarray*}
for suitable $\alpha$, from which it follows that $\tilde{\pi}$ is a Normal
density.  In view of the constraints, it must be the density for
$N(\mu,\sigma^2$).
\vskip 4mm
\ni\underline{\bf Example 2:}  Suppose $\Theta$ and $\theta$ are as in
the previous example, but now only the prior mean of $\theta$ is
specified.  Then $\tilde{\pi}$ must be of the form
$$\tilde{\pi} (\theta)={\exp(\lambda_1 \theta) \over
  \int_{-\infty}^\infty \exp (\lambda_1 \theta) d\theta}.$$
No such distribution exists (since the integral is infinite).
\vskip 12mm
\ni\underline{\bf Quotes} (on the use of subjective priors):
\vskip 4mm
\ni{\it George Box}: ``In the past, the need for probabilities expressing
prior belief has often been thought of, not as a necessity for all
scientific inference, but rather as a feature peculiar to Bayesian
inference.  This seems to come from the curious idea that an outright
assumption does not count as a prior belief.  $\ldots$ I believe that it is
impossible logically to distinguish between model assumptions and the
prior distribution of the parameters.''
\vskip 4mm
\ni{\it I.J.Good}: ``The subjectivist states his judgements, whereas the
objectivist sweeps them under the carpet by calling assumptions
knowledge, and he basks in the glorious objectivity of science.''
\vskip 6mm



\subsection{Bayesian Robustness}

 In some cases, the form chosen for the prior can be important,
  in the sense that different priors can lead to quite different
  posteriors (even if, for example, the priors match in certain ways, such as the mean and
  certain percentiles).   In many applications of Bayesian techniques, it is thus
  important to assess the sensitivity of the conclusions to aspects of
  the prior.  In such analyses it makes sense to identify a range of ``reasonable" priors,
  being priors that you think might reflect opinions of other reasonable individuals (e.g. readers of an article), even  if these do not  actually reflect your own opinions. In particular, if trying to make
  a particular case, it can be helpful to consider priors that are skeptical regarding that
  case, and investigate whether the evidence in the data is enough to overwhelm such skepticism.
 
 Note: if different reasonable priors lead to substantively different conclusions, then
 you have learned something very important: that what one should believe after viewing
 the data depends on ones beliefs before viewing the data, or in other words that the
 data are not sufficiently informative to draw strong conclusions. It is important to note
 that {\it this indicates a fundamental limitation of the data}, and {\it not} a limitation of the Bayesian
 approach. In particular, it is not a sign that one should discard the Bayesian analysis and look
 for a different analysis that leads to a more concrete or confident conclusion! It is often
 the case that a careful Bayesian analysis will result in more nuanced or more uncertain conclusions.
 This can be a hard sell to collaborators who want to publish strong conclusions- but it is an 
 element that should be embraced, as reducing the risk of publishing {\it incorrect} strong
 conclusions!
 
\underline{Toy Example of prior affecting posterior}:  Suppose that $x\sim N(\theta,1)$ and that the
prior median of $\theta$ is 0, the first quartile is -1 and the third
quartile is +1.  If the prior is taken to be normal, we must have
$\theta \sim N(0,2.14)$.  If the prior is Cauchy, it must be C(0,1)
(with pdf $1/(\pi(1+\theta^2))$).  

Suppose $x=4$, an observation
quite compatible with the prior information in both cases.  The,
posterior mean for $\theta$ takes the value 2.75 in the Normal analysis and 3.76 in the
Cauchy prior analysis.

Note that, in general, sensitivity on the prior can depend on the data. Here are the posterior means
for different values of $x$.
\begin{center}
\begin{tabular}{|c|c|c|c|c|c|}\hline
Observed $x$ & 0 & 1 & 2 & 4.5 & 10 \\ \hline \hline
Posterior mean for Normal Prior & 0 & 0.69 & 1.37 & 3.09 & 6.87 \\ \hline
Posterior mean for Cauchy Prior & 0 & 0.55 & 1.28 & 4.01 & 9.80 \\ \hline
\end{tabular}
\end{center}

\subsubsection{Contamination priors}

For a more formal robustness analysis, one could specify a class of
priors and calculate the extent to which aspects of the posterior
change for priors from the class.  For details of natural candidate
classes, see Robert \S3.5, or for more details, Berger \S4.7.
\vskip 4mm
\ni\underline{\bf Example:}  (Berger, example 26, page 212).  Suppose
$x\sim N(\theta,\sigma^2)$, with $\sigma^2$ known.  Consider initially
$\pi_0$, a $N(\mu,\tau^2)$ prior for $\theta$.  Define an alternative
collection of priors
$$
Q=\{ q_k: q_k \mbox{ is a uniform } (\mu-k,\mu+k)
\mbox{ density}\}.
$$  Now consider the class
$$\Gamma=\{\pi :\pi=(1-\varepsilon)\pi_0 + \varepsilon q, q\in Q\},
$$
called an $\varepsilon$-contamination class of priors.  ($\Gamma$
consists of prior beliefs that with probability $1-\varepsilon$, the
prior is $\pi_0$, but otherwise it is $q\in Q$.)

Now consider a fixed
interval $C=(c_1,c_2)$ and ask about the range of values of
$$P_{\pi(\theta \mid x)} (\theta \in C),\ \ \pi \in \Gamma,
$$
the posterior probability that $\theta \in C$ as the prior, $\pi$,
ranges over $\Gamma$.

It follows from Bayes Theorem, that for
$\pi=(1-\varepsilon)\pi_0 + \varepsilon q_k$, where $q_k \in Q$,
$$P_{\pi(\theta \mid x)} (\theta \in C)=\lambda_k (x) P_0 +
(1-\lambda_k (x)) Q_k$$
where 
$$\lambda_k(x)=\left( 1+ \frac{\varepsilon}{1-\varepsilon} \frac{p(x
    \mid q_k)}{p(x \mid \pi_0)} \right)^{-1},$$
$p(x\mid \cdot)$
is the predictive density for $x$ when the prior is $\cdot$, and $P_0$
and $Q_k$ are 
$P_{\pi_0 (\theta \mid x)} (\theta \in C)$ and $P_{q_k (\theta \mid x)}
(\theta \in C),$
the posteriors corresponding to priors $\pi_0$ and $q_k$ respectively.
(Thus, note that when the prior is a mixture, the posterior is a
mixture of the appropriate posteriors with the mixing weights
depending on the data.)

Here the predictive density $p(x\mid \pi_0)$
is $N(\mu,\sigma^2 + \tau^2)$,
$$p(x\mid q_k)=\int_{\mu+k}^{\mu+k} \psi \left(
  \frac{x-\theta}{\sigma} \right) \frac{1}{2k} d\theta,$$
where $\psi$ is standard Normal pdf, and
\begin{eqnarray*}
Q_k&=&P_{q_k(\theta \mid x)}(\theta \in C)\\
&=&\frac{1}{p(x \mid q_k)} \int_{c^\star}^{c^{\star \star}} \psi
\left( \frac{x-\mu}{\sigma} \right) \frac{1}{2k} d\theta,
\end{eqnarray*}
where $c^{\star}=\max\{ c_1,\mu-k\},c^{\star \star}=\min\{ c_2,\mu+k\}$.

As a
concrete example, consider
$\varepsilon=0.1,\sigma^2=1,\tau^2=2$,$\mu=0,x=1$, and
$c=\{-0.93,2.27\}$.  For these values, $C$ is the 95\% credible region
for the prior $\pi_0$.  Then
$$\inf_{\pi \in \Gamma}{P_{\pi(\theta \mid x)} (\theta \in C)} =0.945,$$
acheived at $k=3.4$, and
$$\sup{\pi \in \Gamma}{P_{\pi(\theta \mid x)} (\theta \in C)} =0.956,$$
acheived at $k=0.93$.

Thus, at least for the class $\Gamma$, the
statement that the posterior probability that $\theta \in C$ is 0.95
is very robust.


\end{document}
